
% -------------------------------------------------------
%  Abstract
% -------------------------------------------------------


\pagestyle{empty}

\شروع{وسط‌چین}
\مهم{چکیده}
\پایان{وسط‌چین}
\بدون‌تورفتگی
توسعه دستگاه‌های هوشمند متحرک با بهبود ارتباطات و قابلیت‌های ادراکی موجب تکثیر بسیاری از برنامه‌های کاربردی پیچیده و محاسباتی شده‌است. دستگاه‌های هوشمند با منابع محدود بیش از هر زمان دیگری با محدودیت‌های شدید ظرفیت روبه‌رو هستند. به نظر می‌رسد به عنوان یک مفهوم جدید از معماری شبکه و توسعه محاسبات ابری، محاسبات لبه یک راه‌حل امیدوارکننده برای پاسخ‌گویی به این چالش نوظهور است. محاسبات لبه با نزدیک‌کردن منابع محاسباتی به کاربران نهایی و توزیع محاسبات در سیستم، دستگاه‌ها را از بار سنگین محاسباتی تخلیه می‌کند و موجب بهبود عملکرد سیستم، کیفیت خدمات و کیفیت تجربه کاربران می‌شود. باتوجه به پویایی سطح بار محاسباتی نامشخص در گره‌های لبه، تخصیص‌منابع محاسباتی و تصمیم‌گیری در مورد محل انجام محاسبات، در راستای مرتفع‌کردن نیازهای برنامه‌های کاربردی، امری چالش‌برانگیز خواهدبود. در این پژوهش، وظایف حساس به تأخیر و غیرقابل تقسیم را به همراه پویایی سطح بار محاسباتی سیستم در نظر می‌گیریم، و یک مسئله بارسپاری وظیفه را برای به حداقل رساندن هزینه طولانی‌مدت مورد انتظار فرموله می‌کنیم. در جهت بهبود کارایی سیستم، به بهینه‌سازی مشترک تأخیر و مصرف انرژی می‌پردازیم و یک الگوریتم توزیع‌شده مبتنی بر یادگیری تقویتی عمیق بدون‌مدل را پیشنهاد می‌کنیم، که در آن هر دستگاه می‌تواند تصمیم بارسپاری محاسبات خود را بدون دانستن مدل‌های وظیفه در دستگاه‌های دیگر و سطح بار محاسباتی در گره‌های لبه، تعیین‌کند. برای بهبود برآورد هزینه بلندمدت در  این الگوریتم، از حافظه کوتاه‌مدت ماندگار، شبکه $Q$ عمیق دوئل، و تکنیک‌های شبکه $Q$ عمیق دوگانه بهره می‌گیریم. نتایج ارزیابی نشان می‌دهد که روش پیشنهادی می‌تواند به نحوی کارآمد از ظرفیت‌های محاسباتی گره‌های لبه بهره‌برداری کند و در جهت بهبود کارایی سیستم موثر واقع شود. این روش می‌تواند علاوه بر کاهش چشمگیر میانگین تاخیر و مصرف انرژی در سیستم، به افزایش تعداد وظایف انجام‌شده در مقایسه با چندین روش موجود دست یابد.  


\پرش‌بلند
\بدون‌تورفتگی \مهم{کلیدواژه‌ها}: 
محاسبات لبه، بارسپاری‌محاسباتی، تخصیص‌منابع، مصرف انرژی، یادگیری ‌تقویتی عمیق 
\صفحه‌جدید
