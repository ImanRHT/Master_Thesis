

\فصل{مقدمه}

در این فصل ابتدا پیش‌زمینه‌ای در مورد موضوع پژوهش بیان می‌شود، سپس به بیان انگیزه پژوهش، اهدف و چالش‌های موجود در این حوزه می‌پردازیم.


\قسمت{پیش‌زمینه}

در سال‌های اخیر، با توسعه دستگاه‌های هوشمند سیار، از جمله تلفن‌های همراه هوشمند، وسایل‌نقلیه هوشمند، دستگاه‌های نظارت بر سلامت و سایر موارد، شاهد طیف گسترده‌ای از تعاملات فناوری‌های هوشمند در زندگی روزمره انسانی هستیم. قابلیت‌های ارتباطی و ادراکی این دستگاه‌ها در حال پیشرفت است، در نتیجه شاهد گسترش سریع تعداد زیادی از برنامه‌های کاربردی پیچیده و محاسباتی مانند واقعیت مجازی\پاورقی{Virtual Reality}، مدل‌سازی سه‌بعدی، پردازش زبان طبیعی\پاورقی{Natural Language Processing} و بازی‌های تعاملی هستیم، که در پی آن با تنوع نیاز‌های برنامه‌های کاربردی از جمله نیاز به محاسبات پیچیده و نیاز به پاسخ بی‌درنگ\پاورقی{Real-time}، روبه‌رو می‌شویم. 

با توجه به توان باتری و قدرت محاسباتی محدود دستگاه‌های هوشمند، انجام این محاسبات توسط خود دستگاه به‌تنهایی ممکن نخواهدبود. بنابراین دستگاه‌های متصل به شبکه، راهی به‌جز اتصال به سرورهای ابری برای بارسپاری محاسبات سنگین و استفاده بهینه از منابع محدود خود ندارند. با این حال به‌کارگیری منابع محاسباتی ابری معمول، به علت فاصله مکانی زیاد آن‌ها با کاربران منجر به تأخیر انتقال زیادی می‌شود، که در بسیاری از کاربردهای کنونی از جمله کاربردهایی که به تضمین بی‌درنگ بودن نیاز دارند، مشکل ایجاد می‌کند.

دستگاه‌های هوشمند با منابع محدود بیش از هر زمان دیگری با محدودیت‌های شدیدی در ظرفیت روبه‌رو هستند، و این درحالی است که اجرای تمام برنامه‌ها در سرورهای محاسباتی ابری علاوه بر تأخیر انتقال، به راحتی منجر به ازدحام شبکه و تخریب عملکرد جدی می‌شود. در این زمان، به نظر می‌رسد که به عنوان یک مفهوم جدید از معماری شبکه و توسعه محاسبات ابری\پاورقی{Cloud Computing}، محاسبات لبه\پاورقی{Edge computing} یک راه حل امیدوارکننده برای پاسخگویی به این چالش نوظهور باشد. محاسبات لبه با نزدیک کردن منابع محاسباتی و ذخیره‌سازی به کاربران انتهایی نه تنها می‌تواند از ازدحام شبکه اصلی بکاهد بلکه با برآورده ساختن نیازهای سختگیرانه تأخیر در  عملکرد سیستم و کیفیت تجربه مشتری نیز تا حد زیادی بهبود می‌بخشد. 

جای‌گیری سرورهای محاسباتی در لبه شبکه و استفاده از محاسبات لبه‌ سیار\پاورقی{Mobile Edge computing}، ظرفیت‌های محاسباتی جدیدی را برای کاربران در لبه شبکه بی‌سیم فراهم می‌کند. که علاوه بر ایجاد منابع محاسباتی برای کاربران، تأخیر انتقال را نیز به شدت کاهش می‌دهد و باعث افزایش کیفیت خدمات دریافتی کاربر می‌شود. به این شکل قدرت محاسباتی دستگاه‌های موجود در لبه شبکه با بهره‌گیری از پیشرفت‌های اخیر محاسبات لبه سیار، به مقدار قابل توجهی افزایش می‌یابد~\cite{mao2016dynamic}.

معماری محاسبات لبه دارای مزایای مهمی در رسیدگی به تکثیر دستگاه‌های هوشمند و برنامه‌های کاربردی است، با این حال محدودیت‌ها و چالش‌های زیادی را نیز در بر خواهد‌داشت. از جمله این محدودیت‌ها می‌توان به هزینه بالای استقرار و نگهداری زیرساخت‌ و سرورهای قدرتمند ایستگاه‌های پایه‌ای در لبه شبکه اشاره‌کرد، که هزینه غیرقابل قبولی درپی خواهد‌داشت. و همچنین با اینکه ظرفیت‌های محاسباتی و ذخیره‌سازی سرورهای لبه به مراتب بیشتر از تجهیزات موجود در دستگاه‌های هوشمند می‌باشد، اما تعداد محدودی از برنامه‌ها را می‌توان در یک سرور لبه برای ارائه خدمات به دستگاه‌های هوشمند متعدد مستقرکرد. 

به عنوان مهم‌ترین چالش در محاسبات لبه می‌توان به تنوع فعالیت‌های انسانی در زندگی روزمره اشاره کرد، که باعث ایجاد نیازهای متنوع دستگاه‌های هوشمند و تغییر وضعیت مناطق درخواست شده و موجب بی‌ثباتی سیستم می‌شود. همچنین حجم بالای درخواست‌های پشت سر هم، می‌تواند باعث افزایش ناگهانی بار محاسبات در سرورهای لبه و همچنین عدم تعادل بار جدی در بین آن‌ها شود، که باعث تراکم درخواست‌ها در مناطق خاص و بالا رفتن نرخ تأخیر در پاسخ‌گویی خواهدشد.


با توجه به محدودیت‌ها و چالش‌های موجود در محاسبات لبه، چگونگی تخصیص منابع محاسباتی به شکلی منعطف و مقیاس‌پذیر به یک چالش بزرگ در این حوزه تبدیل شده‌است. بیشتر پژوهش‌های انجام‌گرفته بر تخلیه وظیفه\پاورقی{Task Offloading} محاسباتی، بهینه‌سازی در برنامه‌ریزی\پاورقی{Scheduling} محاسبات و تخصیص منابع\پاورقی{Resource allocation} محاسباتی‌لبه تمرکز دارند، و مسئله بارسپاری محاسباتی را معمولاً به علت وجود متغیرهای دوحالتی تصمیم‌گیری به صورت برنامه‌ریزی عدد صحیح آمیخته\پاورقی{Mixed Integer Linear Programming}  بیان می‌کنند، و برای حل آن الگوریتم هایی همچون برنامه‌ریزی‌پویا\پاورقی{Dynamic programming} ارائه شده‌است که برای شبکه‌های بزرگ با تعداد بالای سرورها، پیچیدگی زیادی در بر دارد~\cite{bertsekas2012dynamic}. 

اگرچه این پژوهش‌ها زمینه را برای تحقیقات بر روی محاسبات لبه، در جنبه‌های مختلف فراهم می‌کند، اما این مطالعات به سازگاری روش‌های تخصیص منابع در محاسبات لبه در مواجهه با بار سنگین و عدم‌ثبات شبکه نپرداخته‌اند. هنگامی که مناطق تولید درخواست و همچنین نیازهای دستگاه‌های هوشمند به طرز چشمگیری  افزایش یابد، بدون در نظر گرفتن شرایط مختلف، تأثیر قابل توجهی در عملکرد سیستم از جمله توزیع بار محاسباتی خواهدداشت. 

%\صفحه‌جدید

\قسمت{اهداف پژوهش}

در این پژوهش سعی داریم با هدف کاهش تأخیر متوسط محاسبات و بهینه‌سازی مصرف انرژی، به یک سیاست بهینه تخصیص‌منابع، در یک شبکه محاسباتی لبه دست پیداکنیم. ما پویایی سطح بار ناشناخته را در گره‌های لبه در نظر می‌گیریم و یک الگوریتم بارسپاری توزیع‌شده برای سیستم محاسبات لبه متحرک، با بهره‌گیری از یادگیری تقویتی عمیق\پاورقی{Deep Reinforcement Learning} ارائه می‌دهیم. در الگوریتم پیشنهادی، هر دستگاه هوشمند متحرک می‌تواند تصمیم‌گیری تخلیه وظایف را به صورت غیرمتمرکز با استفاده از اطلاعاتی که به صورت محلی مشاهده می‌شود، از جمله اندازه وظیفه، اطلاعات صف‌ها و سطوح بار تاریخی در گره‌های لبه، اخذ کند. 

ما عمل تخلیه وظیفه را برای وظایف غیر قابل تقسیم و حساس به تأخیر\پاورقی{Delay-sensitive} فرموله می‌کنیم، و در جهت دستیابی به حداقل هزینه‌های موردانتظار طولانی‌مدت، تابع هزینه را مطابق با تأخیر محاسبات و مصرف انرژی در نظر می‌گیریم. برای دستیابی به پویایی سطح بار محاسباتی سیستم از یک حافظه کوتاه‌مدت ماندگار\پاورقی{Long Short-term Memory} استفاده می‌کنیم، که با یادگیری اطلاعات تاریخی می‌تواند سطوح بار را در آینده نزدیک پیش‌بینی کند و این سطح بار پیش‌بینی‌شده، قسمتی از مشاهدات محیطی را در شبکه یادگیری تقویتی تشکیل می‌دهد. 

شایان ذکر است که منابع مورد نیاز وظایف محاسباتی در هر شکاف زمانی متفاوت و پویا است، یک دستگاه هوشمند خاص ممکن است به دلیل منابع محاسباتی بیش از حد مورد نیاز تحت محدودیت حداکثر تأخیر مجاز، قادر به اجرای کار وارد شده به صورت محلی نباشد. بنابراین، تفاوت کلیدی بین روش‌ پیشنهادی و روش‌های پایه در این است که روش‌ پیشنهادی می‌توانند تصمیم‌گیری‌های بارسپاری را صورت پویا انجام‌دهد.


%ما از معماری یادگیری تقویتی عمیق برای بهینه‌سازی شبکه و تخصیص منابع محاسباتی در سیستم‌های محاسبات لبه متحرک استفاده می‌کنیم و الگوریتم بارسپاری وظیفه محاسباتی توزیع‌شده را پیشنهاد می‌دهیم. 

%با استفاده از یک شبکه عصبی حافظه کوتاه‌مدت ماندگار به پیش‌بینی سطوح بار محاسباتی می‌پردازیم که در همگرایی و تخمین سیاست بهینه بارسپاری،  می‌تواند بسبار موثر واقع‌شود.  




%\شروع{فقرات}

%\فقره تخلیه محاسباتی وظیفه در محاسبات لبه متحرک: 
%ما عمل تخلیه وظیفه را برای وظایف غیر قابل تقسیم و حساس به تأخیر فرموله می‌کنیم. چالش موجود این است که پویایی سطح بار، در گره‌های لبه را در نظر بگیرد، و هدف آن به حداقل رساندن هزینه‌های مورد انتظار از جمله تأخیر و مصرف انرژی خواهدبود.


%\فقره  الگوریتم تخلیه وظیفه مبتنی بر یادگیری تقویتی عمیق:
%در جهت دستیابی به حداقل هزینه‌های موردانتظار طولانی مدت، و برای بهبود برآورد هزینه‌ها در الگوریتم پیشنهادی، ما از حافظه کوتاه مدت طولانی، شبکه کیو عمیق و تکنیک های شبکه دوئل کیو عمیق استفاده می‌کنیم.


%\پایان{فقرات}





\قسمت{چالش‌ها}



یکی از چالش‌های موجود در مسئله بارسپاری محاسبات، تعیین نحوه انجام این فرایند در لبه شبکه است. یک استراتژی خوب می‌تواند منجر به کاهش انرژی مصرفی در شبکه و همچنین کاهش زمان پاسخ سیستم شود.
 دو چالش اصلی در تخلیه وظایف محاسباتی وجود دارد، اولین چالش این است که یک دستگاه هوشمند متحرک، در چه زمانی و در مواجه با چه نوع وظایفی باید محاسبات خود را به گره‌های لبه تخلیه کند، و چالش دوم این است که، اگر تصمیم بر تخلیه محاسبات باشد، این امر به چه شکل و به کدام گره لبه باید انجام شود. برای رسیدگی به این چالش‌ها، خیل زیادی از پژوهش‌های جدید به توسعه الگوریتم‌های تخلیه محاسباتی پرداخته‌اند.


زمان پردازش یک بار محاسباتی در سرورهای مختلف با توجه به قدرت پردازش آنها متفاوت است. همچنین سرعت انتقال داده بین کاربران و سرورها و بین سرورهای مختلف با توجه به کیفیت متفاوت راه‌های ارتباطی بین آنها تعیین می‌شود. تمام این متغیرها در تأخیر انجام محاسبات تأثیر بسزایی دارد و می‌تواند سیاست تخصیص منابع و فرآیند کاری را کاملاً تغییردهد. این موضوع زمانی که چندین کاربر با محاسبات مختلف داریم چالش بیشتری ایجاد می‌کند. به طور مثال کاربران به طور طبیعی سروری با قدرت محاسباتی بالاتر را انتخاب می‌کند اما در صورتی که کاربران زیادی به این سرور بارسپاری کنند زمان پاسخ آن افزایش‌یافته و دیگر سرور مناسبی برای بارسپاری محاسباتی محسوب نمی‌شود.

بنابراین نحوه تخصیص منابع به کاربران مختلف و پخش بار محاسباتی در شبکه به شکلی که متوسط زمان انجام محاسبات کاهش‌یابد از اهمیت ویژه‌ای برخوردار است.


\صفحه‌جدید

\قسمت{دست‌آوردها}

در این پژوهش به بررسی چالش بارسپاری وظایف در یک سیستم محاسباتی لبه پرداخته و با هدف به حداقل رساندن میانگین تأخیر و مصرف انرژی در طولانی‌مدت یک الگوریتم بارسپاری توزیع‌شده ارائه می‌دهیم، که دستگاه‌های تلفن همراه را قادر می‌سازد تا تصمیم‌گیری بارسپاری وظایف خود را به شیوه‌ای غیرمتمرکز انجام‌دهند، در حالی که می‌توانند پویایی سطح بار ناشناخته را در گره‌های لبه در نظر گرفته و به بهینه‌سازی کارایی سیستم در مواجهه با وظایف غیرقابل تقسیم و حساس به تأخیر بپردازد. و این در حالی است که محدودیت تأخیر و ظرفیت محدود منابع نیز در نظر گرفته شده است. 

نتایج شبیه سازی نشان می‌دهد که در مقایسه با چندین روش موجود، الگوریتم پیشنهادی می تواند نسبت کارهای منقضی شده، میانگین تاخیر و میزان مصرف انرژی را کاهش دهد. این مزیت به ویژه زمانی قابل توجه است که وظایف حساس به تاخیر باشند، یا سطوح بار در گره های لبه بالا باشد.


\قسمت{ساختار پایان‌نامه}
ساختار ادامه پایان‌نامه در این قسمت بیان می‌شود. در  \textbf{فصل ۲}، به بیان ادبیات موضوع و مفاهیم اساسی در حوزه پژوهش می‌پردازیم و در \textbf{فصل ۳}، پژوهش‌های مرتبط پیشین را بررسی و طبقه‌بندی می‌کنیم.

 در \textbf{فصل ۴}، ابتدا به تعریف مسئله در محاسبات لبه پرداخته و ساختار مدل سیستم را تعریف می‌کنیم.  سپس به معرفی مدل دستگاه‌های تلفن همراه و مدل گره‌های لبه می‌پردازیم. مدل دستگاه‌های تلفن همراه را تحت قالب مدل وظیفه، صف‌ محاسبات و صف انتقال مورد بررسی قرار می‌دهیم و مدل گره‌های لبه را در قالب صف محاسبات دستگاه در گره لبه بیان می‌کنیم.   
در \textbf{فصل ۵}، به بررسی چالش بارسپاری وظیفه در محاسبات لبه، بر مبنای یادگیری تقویتی عمیق می‌پردازیم. مسئله را در چهارچوب مولفه‌های یادگیری تقویتی بیان‌کرده و شبکه‌های عصبی استفاده‌شده در الگوریتم پیشنهادی را تشریح می‌کنیم. الگوریتم پیشنهادی را معرفی‌کرده و نحوه استقرار آن در دستگاه‌های تلفن همراه و گره‌های لبه را بیان می‌کنیم. سپس در \textbf{فصل ۶} به تشریح شبیه‌سازی، ارزیابی عملکرد و مقایسه نتایج حاصل از آن می‌پردازیم. و در \textbf{فصل ۷}، به نتیجه‌گیری پرداخته و پژوهش‌های آینده را معرفی می‌کنیم.





































