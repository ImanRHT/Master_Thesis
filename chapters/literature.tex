
\فصل{پژوهش‌های مرتبط پیشین}

در این فصل پژوهش‌های مرتبط پیشین در این حوزه را مورد بررسی قرار می‌دهیم.

\قسمت{مقدمه}

در مسائل مربوط به تخصیص منابع در محاسبات لبه‌ای و ابری، منابع محاسباتی مانند پردازشگرها، حافظه و پهنای‌باند باید به کاربران تخصیص داده‌شود. با توجه به محدود بودن این منابع در سرورهای لبه شبکه و فاصله زیاد سرورهای ابری، تخصیص منابع به منظور توزیع بهینه منابع محدود و همچنین بهبود عملکرد سیستم از اهمیت ویژه‌ای برخوردار است. از جمله محدودیت‌هایی که برای تخصیص منابع وجود دارد کمبود منابع موجود، محدودیت‌های انرژی مصرفی و تأخیر است.



ایده اصلی در پژوهش‌های پیشین، تخصیص منابع به منظور بهینه‌سازی یک تابع هدف است، که این بهینه‌سازی معمولاً کمینه‌کردن تأخیر یا انرژی مصرفی و یا ترکیب خطی از هر دوی آن‌ها است. همچنین پژوهش‌های پیشین را می‌توان از این منظور که تخصیص‌دهنده منابع و تصمیم‌گیرنده بارسپاری به صورت متمرکز یا غیر متمرکز  عمل می‌کند، طبقه‌بندی کرد.

محوریت پژوهش‌های موجود پیرامون تخصیص منابع در محاسبات لبه سیار، تخلیه و بارسپاری محاسبات می‌باشد. دو سوال اصلی در مورد بارسپاری محاسبات وجود دارد. اولین سوال این است که یک دستگاه هوشمند سیار، در مواجهه با شرایط مختلف باید محاسبات خود را به یک گره لبه بارسپاری‌کند یا به صورت محلی انجام‌دهد. سوال دوم این است که اگر یک دستگاه تصمیم به بارسپاری محاسبات گیرد، آنگاه باید محاسبات خود را به کدام یک از گره‌های لبه تخلیه‌کند. برای پرداختن به این سوالات، برخی از آثار موجود، الگوریتم‌های بارسپاری محاسبات را پیشنهاد کرده‌اند. 


%%در بیشتر پژوهش‌ها، مدل ارائه‌شده برای محاسبات، محاسبه را به صورت یک واحد در نظر می‌گیرد، که دارای حجم داده مشخصی برای انتقال و مقدار پردازش معینی برای محاسبه است. اخیراً در مدل‌های واقع‌گرایانه‌تر برای یک محاسبه چندین زیر محاسبه در نظر گرفته شده است که بین زیر محاسبه‌ها وابستگی وجود دارد. در این حالت محاسبات را معمولاً به صورت گراف جهت دار مدل می کند مدل های گراف نشان دهنده وابستگی بین‌ زیر محاسبات است. در ادامه به تعدادی از کارهای پیشین که به روش ارائه شده در این پایان‌نامه نزدیک تر است اشاره می کنیم.%%


\قسمت{پژوهش‌های پیشین}

پژوهش‌های بسیاری وجود دارد که مسئله تخصیص منابع محاسباتی در لبه شبکه را به صورت مسئله برنامه‌ریزی عددصحیح آمیخته مدل می‌کنند. به عنوان مثال در~\cite{tran2018joint} بعد از مدل کردن مسئله به صورت برنامه‌ریزی عدد صحیح آمیخته یک الگوریتم جستجوی ابتکاری را برای بارسپاری چندین محاسبه در یک شبکه شامل چند سرور لبه سیار ارائه می‌دهد، و به صورت بازگشتی متغیرهای دو حالتی مربوط به تصمیم‌گیری بارسپاری را به‌دست می‌آورد.

یکی دیگر از روش‌هایی که در بسیاری از پژوهش‌ها استفاده شده‌است، ضعیف کردن شرایط مسئله برای رسیدن به یک مسئله بهینه‌سازی محدب است. این کار معمولاً با تضعیف شرط دوحالتی بودن متغیر‌های تصمیم‌گیری به متغیرهای پیوسته بین صفر و یک انجام می‌شود~\cite{guo2016energy}. همچنین در~\cite{sundar2018offloading} علاوه‌بر ضعیف‌کردن شرط دوحالتی متغیرهای تصمیم‌گیری، توان محاسباتی سرور ابری، بی‌نهایت در نظر گرفته شده‌است.

در~\cite{sun2017latency} بار سپاری به صورت یک مسئله بهینه‌سازی با تابع هدف تأخیر متوسط مدل شده‌است، و یک الگوریتم برای حل کردن موثر مسئله بهینه‌سازی ارائه شده‌است. در این مقاله ورود بارهای محاسباتی که به صورت واحد مدل شده‌اند به صورت پواسون\پاورقی{Poisson} در نظر گرفته شده‌است. در نهایت برای رسیدن به متوسط بهینه تأخیر محاسبات، یک مسئله بهینه‌سازی غیر چندجمله‌ای\پاورقی{Polynomial} بیان‌شده و یک الگوریتم ابتکاری برای به‌دست‌آوردن سیاست بهینه بارسپاری ارائه شده‌است.




نویسندگان در~\cite{wang2017computation} الگوریتمی را برای تعیین تصمیمات تخلیه دستگاه‌های تلفن همراه در جهت به حداکثر رساندن بازده شبکه پیشنهاد کرده‌اند. در~\cite{bi2018computation} بر روی یک سناریوی محاسبات لبه متحرک بی‌سیم تمرکز کرده‌اند و الگوریتمی برای بهینه‌سازی مشترک تصمیمات تخلیه و انتقال نیرو پیشنهاد کرده‌اند. در آثار~\cite{wang2017computation} و~\cite{bi2018computation}، ظرفیت محاسباتی موجود در گره‌های لبه برای هر دستگاه تلفن همراه، مستقل از تعداد وظایف تخلیه‌شده‌ی آن دستگاه به گره لبه درنظر گرفته شده‌است.




در عمل گره‌های لبه ممکن است ظرفیت پردازش محدودی داشته‌باشند، بنابراین ظرفیت پردازشی که یک گره لبه به دستگاه تلفن همراه اختصاص می‌دهد به سطح بار در گره لبه بستگی دارد. هنگامی که تعداد زیادی از دستگاه‌های تلفن همراه وظایف خود را در یک گره لبه به‌خصوص تخلیه کنند، بار محاسباتی در آن گره لبه می‌تواند بیش‌ازحد افزایش یابد و از این‌رو آن وظایف بارگذاری‌شده ممکن است با تأخیر پردازش زیادی مواجه شوند. حتی ممکن است برخی از وظایف با اتمام مهلت آنها، منقضی شوند.

برخی از پژوهش‌های موجود به سطوح بار محاسباتی در گره‌های لبه پرداخته‌اند و الگوریتم‌های تخلیه وظیفه متمرکز را پیشنهاد کرده‌اند. نویسندگان در~\cite{eshraghi2019joint} الزامات محاسباتی نامشخص دستگاه‌های تلفن همراه را در نظر گرفته‌‌اند و الگوریتی را پیشنهادکرده‌اند که تصمیمات تخلیه دستگاه‌های تلفن همراه و تخصیص منابع محاسباتی گره لبه را بهینه می‌کند. در~\cite{lyu2018energy} بر روی وظایف حساس به تأخیر متمرکز شده‌اند و الگوریتمی را برای به حداقل رساندن مصرف انرژی تخلیه محاسبات با توجه به محدودیت مهلت وظیفه پیشنهاد کرده‌اند. در~\cite{chen2018task} یک شبکه فوق متراکم تعریف‌شده توسط نرم‌افزار در نظر گرفته‌شده و یک الگوریتم متمرکز برای به حداقل رساندن تأخیر پردازش وظیفه طراحی کرده‌اند. به همین شکل در~\cite{poularakis2019joint} بهینه‌سازی مشترک تخلیه و مسیریابی وظیفه را با درنظرگرفتن الزامات نامتقارن وظایف، مورد مطالعه قرار داده‌اند. با این حال، این الگوریتم‌های متمرکز در~\cite{eshraghi2019joint} -~\cite{poularakis2019joint} ممکن است به اطلاعات کلی سیستم (به عنوان مثال، ورود و اندازه وظایف همه دستگاه‌های تلفن همراه) نیاز داشته‌باشند و سربار بالایی داشته باشند.




آثاری دیگر الگوریتم‌های تخلیه وظیفه توزیع‌شده را با درنظرگرفتن سطوح بار محاسباتی در گره‌های لبه پیشنهاد کرده‌اند، به شکلی که هر دستگاه تلفن همراه تصمیم تخلیه وظیفه خود را به شیوه‌ای غیرمتمرکز می‌گیرد. طراحی چنین الگوریتم توزیع‌شده‌ای چالش‌های زیادی به همراه دارد، به این دلیل که وقتی یک دستگاه تصمیم به تخلیه وظیفه می‌گیرد، وضعیت  سطوح بار محاسباتی در گره‌های لبه را نمی‌داند، چراکه سطوح بار، به تصمیم‌گیری‌های تخلیه و مدل‌های وظیفه (مثلاً اندازه و زمان رسیدن) در سایر دستگاه‌های تلفن همراه نیز بستگی دارد. 

برای رسیدگی به این چالش‌ها، نویسندگان در~\cite{lyu2018distributed} بر روی وظایف قابل تقسیم متمرکز شده‌اند و یک الگوریتم مبتنی بر لیاپانوف\پاورقی{lyapunov optimization} را برای اطمینان از پایداری صف‌های وظیفه پیشنهاد کرده‌اند. در~\cite{li2019incentive} تعامل بارگذاری استراتژیک بین دستگاه‌های تلفن همراه را در نظر گرفته و یک الگوریتم توزیع‌شده مبتنی بر هزینه پیشنهاد کرده‌اند. نویسندگان در~\cite{shah2018hierarchical} یک الگوریتم بارگذاری بالقوه مبتنی بر نظریه‌بازی طراحی کرده‌اند تا کیفیت تجربه هر دستگاه را به‌حداکثر برساند. در~\cite{yang2018distributed} یک الگوریتم تخلیه توزیع‌شده برای رسیدگی به رقابت کانال‌های بی‌سیم در بین دستگاه‌های تلفن همراه پیشنهاد کرده‌اند. در~\cite{neto2018uloof} یک روش مبتنی بر برآورد، پیشنهاد کرده‌اند که در آن هر دستگاه تصمیم‌گیری تخلیه خود را بر اساس ظرفیت‌های محاسبات و قدرت انتقال خود انجام می‌دهد.



در این کار، ما بر چالش‌های موجود در تخلیه وظایف محاسباتی در یک سیستم محاسبات لبه متحرک تمرکز می‌کنیم و یک الگوریتم توزیع‌شده را پیشنهاد می‌کنیم که به شکلی منعطف، وظایف محاسباتی ناشناخته را در گره‌های لبه قرار می‌دهد. به شکلی که در مقایسه با پژوهش‌های فوق‌الذکر~\cite{lyu2018distributed} -~\cite{neto2018uloof}، ما یک سناریوی محاسبات لبه متحرک متفاوت و واقع گرایانه را در نظر می‌گیریم. در پژوهش~\cite{lyu2018distributed} وظایف محاسباتی قابل توجهی در نظر گرفت شده‌است، اما از آنجایی که وظایف می‌توانند به صورت خودسرانه تقسیم شود، ممکن است به دلیل وابستگی بین بخش‌های تقسیم‌شده، این امر به دور از یک شرایط واقع‌گرایانه تلقی شود. 

اگرچه آثار ~\cite{li2019incentive} -~\cite{neto2018uloof}، وظالف محاسباتی را به عنوان وظایف غیرقابل تقسیم در نظر گرفته‌اند، اما این سیستم‌ها، صف‌های اساسی را در نظر نمی‌گیرند. به عنوان یک نتیجه، محاسبات و انتقال هر وظیفه، همواره باید در یک واحد زمانی انجام شود، که در عمل ممکن است این امر همیشه تضمین نشود. متفاوت از این آثار ~\cite{lyu2018distributed} -~\cite{neto2018uloof}، ما وظایف غیرقابل تقسیم را با سیستم‌های صف‌بندی در نظر می‌گیریم و سناریوی عملی را که در آن محاسبات و انتقال یک وظیفه می‌تواند برای چندین واحد زمان ادامه یابد را گسترش می‌دهیم. این سناریو  چالش برانگیز خواهدبود، زیرا زمانی که محاسبات جدید وارد می‌شوند، تأخیر آن‌ها را می‌توان تحت‌تأثیر تصمیمات وظایف دستگاه‌های دیگر قرار داد.

متفاوت از کارهای مرتبط ~\cite{lyu2018distributed} -~\cite{neto2018uloof} که وظایف متحمل تأخیر را در نظر می‌گرفتند، ما وظایف حساس به تأخیر را با مهلت‌های زمانی پردازش درنظر می‌گیریم. پرداختن به این موضوع بسیار جذاب است، زیرا مهلت‌های پردازش بر پویایی سطح بار محاسباتی در گره‌های لبه و در نتیجه بر تأخیر وظایف بارگذاری‌شده تأثیر می‌گذارند.


همچنین میزان تحمل‌پذیری تأخیر در وظایف حساس به تأخیر را می‌توان چالشی اساسی در سناریوهای عملی در نظر گرفته‌شده دانست، زیرا مهلت‌های وظایف می‌توانند سطح بار محاسبات در گره‌های لبه را تحت تاثیر قرار دهند، و از این‌رو موجب تأخیر وظایف تخلیه‌شده شوند. تحت سیستم محاسبات لبه متحرک فوق‌الذکر، استفاده از روش‌های سنتی مانند نظریه‌بازی و بهینه‌سازی آنلاین به دلیل تعامل پیچیده در میان وظایف، دشوار است. برای رسیدگی به این چالش‌ها، تکنیک‌های یادگیری تقویتی عمیق (به عنوان مثال یادگیری $Q$ عمیق~\cite{mnih2015human}) روش‌های مناسبی در مواجهه با چالش تخلیه وظیفه در سیستم محاسبات لبه متحرک تلقی می‌شوند، زیرا این روش‌ها عوامل را قادر می‌سازند تا بر اساس مشاهدات محلی و بدون اطلاعات محیط تصمیم‌گیری نمایند.

برخی از آثار موجود مانند~\cite{huang2019deep} -~\cite{liu2019deep}، از الگوریتم‌های مبتنی بر یادگیری تقویتی عمیق در سیستم محاسبات لبه متحرک به شکل متمرکز بهره گرفته‌اند. در حالی که آنها بر روی الگوریتم‌های تخلیه متمرکز، تمرکز کرده‌اند، نویسندگان در~\cite{zhao2019deep} یک الگوریتم تخلیه توزیع‌شده مبتنی بر یادگیری تقویتی عمیق را پیشنهاد کرده‌اند، که به رقابت کانال‌های بی‌سیم بین دستگاه‌های تلفن همراه می‌پردازد، و این در حالی است که الگوریتم پیشنهادی در هر دستگاه تلفن همراه به اطلاعات کیفیت خدمات در سایر دستگاه‌های تلفن همراه نیاز دارد. 



نویسندگان در~\cite{xiong2020resource} یک رویکرد مبتنی بر یادگیری تقویتی عمیق را برای تخصیص‌منابع در محاسبات لبه پیشنهاد کرده‌اند. این پژوهش مسئله تخصیص منابع را به عنوان یک فرآیند تصمیم‌گیری مارکوف فرموله می‌کند، و همچنین یک الگوریتم شبکه $Q$ عمیق بهبود‌یافته را برای یادگیری خط‌مشی سیستم پیشنهاد می‌کند، که در آن حافظه‌ای برای ذخیره جداگانه تجربیات و تأثیر متقابل اعمال در نظر گرفته می‌شود. نتایج شبیه‌سازی نشان می‌دهد که الگوریتم پیشنهادی از نظر هم‌گرایی از الگوریتم اصلی شبکه $Q$ عمیق بهتر عمل می‌کند و سیاست مربوطه بهتر از سایر سیاست‌ها در مورد زمان تکمیل وظیفه عمل می‌کند.


در~\cite{gai2018optimal} نویسندگان به موضوع تخصیص منابع در اینترنت اشیا متمرکز شده‌اند و از سطح رضایتمندی تجربه مشتری برای دستیابی به خدمات محتوامحور\پاورقی{Content-centric Services} استفاده می‌کنند. در این پژوهش با بهره‌گیری از مکانیسم یادگیری‌ تقویتی در راستای دست‌یافتن به عملکرد بهینه در تخصیص منابع، یک رویکرد جدید ارائه شده است. این کار ترکیبی از کیفیت تجربه مشتری و یادگیری تقویتی را برای ایجاد جداول نگاشت هزینه‌ی از پیش ذخیره‌شده استفاده می‌کند. مقادیر جدول هزینه توسط بازخورد از طرف کاربر و از طریق معیار کیفیت تجربه مشتری تعیین می‌شود. تابع ارزش در یادگیری تقویتی از سطح تجربه مشتری به عنوان پاداش برای به‌روزرسانی تصمیمات استفاده می‌کند. این مطالعه به پیاده‌سازی شبکه‌ای محتوامحور برای تحقق تخصیص منابع بهینه تأکید دارد. تابع هزینه مورد استفاده در یادگیری، هم هزینه انرژی و هم تأخیر پاسخ را در نظر می‌گیرد، که قابل توسعه و گسترش برای سایر برنامه‌ها است.  


در~\cite{li2018deep} یک سیستم محاسبات لبه چندکاربره در نظر گرفته شده‌است، که در آن تجهیزات کاربر می‌توانند بارگیری محاسباتی را از طریق کانال‌های بی‌سیم به یک سرور لبه انجام‌دهند. جمع هزینه تأخیر و مصرف انرژی به عنوان هدف بهینه‌سازی در نظر گرفته شده‌اند. در این پژوهش به طور مشترک بارگیری و تخصیص منابع محاسباتی برای بهینه‌سازی تعیین شده‌اند. به همین منظور نویسندگان در این مطالعه از یادگیری تقویتی عمیق در جهت کشف سیاست بهینه استفاده کرده‌اند. 

یادگیری تقویتی یک هدف بلند مدت را در نظر می‌گیرد، که برای سیستم‌های پویا از نوع محاسبات لبه متحرک چندکاربره بسیار مورد توجه است. به همین شکل در~\cite{shan2020drl+} مدلی هوشمند برای تخصیص بهینه منابع بر اساس چارچوب یادگیری تقویتی عمیق برای تخصیص تطبیقی منابع شبکه و نیاز های محاسباتی طراحی شده‌است. این مدل برای آموزش عوامل یادگیری تقویتی عمیق به صورت توزیع‌شده، چارچوب یادگیری انجمنی\پاورقی{Federated Learning} را با محاسبات لبه متحرک تلفیق می‌کند. این مدل به خوبی می‌تواند مشکلات بارگذاری حجم زیاد داده، محدودیت‌های شرایط ارتباطی و حریم خصوصی داده‌ها را حل‌کند. نتایج تجربی نشان می‌دهد که مدل پیشنهادی از الگوریتم‌های سنتی تخصیص منابع با بهره‌گیری تنها از یادگیری تقویتی عمیق از سه جنبه‌ی به حداقل‌رساندن متوسط مصرف انرژی سیستم، به‌حداقل رساندن تأخیر متوسط خدمات و تعادل تخصیص منابع برتر است.






در پژوهش~\cite{liu2020resource} به بررسی تجمع متمرکز کاربران برای گروه‌بندی کاربران اینترنت اشیا در خوشه‌های مختلف بر اساس اولویت‌های کاربر پرداخته شده‌است. خوشه با بالاترین اولویت بار‌سپاری می‌شود و بر روی سرور لبه اجرا می‌شود، در حالی که خوشه با کمترین اولویت محاسبات را به صورت محلی انجام می‌دهد. برای خوشه‌های دیگر، توسعه سیاست‌های جابه‌جایی توزیع‌شده کاربران از طریق فرآیند تصمیم‌گیری مارکوف مدل‌سازی می‌شود، که در آن هر کاربر اینترنت اشیا به‌عنوان عاملی در نظر گرفته می‌شود که مجموعه‌ای از تصمیم‌های جابه‌جایی را در محاسبه هزینه‌های سیستم بر اساس پویایی محیط اتخاذ می‌کند. برای پرداختن به نفرین ابعاد، از یک شبکه $Q$ عمیق شامل یک شبکه عصبی عمیق برای تقریب تابع $Q$ در یادگیری $Q$ استفاده شده‌است.



\شروع{لوح}[t]
\تنظیم‌ازوسط

\شروع{جدول}{|c|c||c|c||c|c|c|}
\خط‌پر
منبع & سال انتشار & تاخیر & انرژی & مهلت & فرآيند تصمیم‌گیری مارکوف & یادگیری تقویتی عمیق \\ 
\خط‌پر \خط‌پر
~\cite{yang2018distributed} & $2018$ & $\surd$ & $\surd$ &  &  &  \\ 

\خط‌پر
~\cite{neto2018uloof} & $2018$ & $\surd$ & $\surd$ &  &  &  \\ 

\خط‌پر
~\cite{huang2019deep}  & $2019$ &  & $\surd$ &  &  & $\surd$ \\ 

\خط‌پر

~\cite{liu2019deep} & $2019$ & $\surd$ & &  & $\surd$ & $\surd$ \\ 


\خط‌پر
~\cite{xiong2020resource}  & $2020$ & $\surd$ & &  & $\surd$ & $\surd$ \\ 

\خط‌پر

~\cite{shan2020drl+} & $2020$ & $\surd$ & $\surd$ &  & $\surd$ & $\surd$ \\ 

\خط‌پر
~\cite{liu2020resource}  & $2020$ & $\surd$ & $\surd$ &  & $\surd$ & $\surd$ \\ 
\خط‌پر

~\cite{tang2020deep}  & $2020$ & $\surd$ &  & $\surd$ & $\surd$ & $\surd$ \\ 

\خط‌پر

~\cite{zhou2021deep}   & $2021$ &  & $\surd$ &  &  & $\surd$ \\ 

\خط‌پر

\پایان{جدول}

\شرح{جمع‌بندی پژوهش‌های مرتبط}
\برچسب{جدول:جمع‌بندی}
\پایان{لوح}


بنابراین با توجه به پژوهش‌های مرتبط پیشین بررسی‌شده، می‌توان دریافت  که در بسیاری از کارها جهت ارزیابی و بهبود عملکرد سیستم به معیارهای تاخیر و انرژی پرداخته شده‌است. علاوه بر این در برخی از پژوهش‌ها مشاهده می‌شود که به عنوان یک محدودیت اصلی در سیستم به تعریف وظایف با محدودیت زمان اجرا توجه شده‌است. از این رو در این پژوهش سعی‌شده است که معیارهای اصلی عملکرد و محدودیت مهلت زمانی برای وظایف غیر قابل تقسیم در نظر گرفته شود. همچنین با تحقیق و بررسی روش‌های به کار گرفته شده جهت بهبود عملکرد در کارهای پیشین به توجه فراوان به فرایند تصمیم‌گیری مارکوف به عنوان یک ابزار بسیار قدرتمند جهت مدل‌سازی سیستم و در پی آن روش‌های گوناگون یادگیری تقویتی برای بهینه‌سازی مدل، بر می‌خوریم. این رویکرد در بسیاری از پژوهش‌های اخیر، کارامد و پویا ارزیابی شده‌است. مقایسه معیارهای مورد بررسی در کارهای پیشین در جدول~\رجوع{جدول:جمع‌بندی} ارائه شده‌است.  


